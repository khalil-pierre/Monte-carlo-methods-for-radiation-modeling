\documentclass[twocolumn,prl,nobalancelastpage,aps,10pt]{revtex4-1}
%\documentclass[rmp,preprint]{revtex4-1}
\usepackage{graphicx,bm,times}

\begin{document}

\title{Monte Carlo methods and }

\author{A.N. Other Student}

\affiliation{Level 3 MSci. Laboratory, Department of Physics, University of Bristol.}

\begin{abstract} 

\end{abstract}
\date{\today}

\maketitle

\section{INTRODUCTION}




\section{EXPERIMENTAL DETAILS}



\section{RESULTS}


\begin{figure}
\includegraphics*[width=0.96\linewidth,clip]{figure1}
\caption{Fractional change in length of the rod ($\Delta L/L$) versus temperature change ($\Delta T$). The solid line
is a fit to the linear relation Eq. (\ref{Expansionequation}). Note that the graph does not need a title as this information is contained in the figure caption.} \label{expansionfigure}
\end{figure}





\section{DISCUSSION}



\section{CONCLUSIONS}



\section{APPENDIX}



%\section{REFERENCES}


%\begin{thebibliography}{99}
%\bibitem{white73} \textit{Thermal expansion of reference materials: copper, silica and silicon}, G. K. White,  Journal of Physics D: Applied. Physics  \textbf{6}, 2070 (1973).

%\bibitem{epr} \textit{Can quantum-mechanical description of physical reality be considered complete?}, A. Einstein, B. Podolsky, N. Rosen, Physical Review \textbf{47}, 0777, (1935).

%\bibitem{feynman} \textit{Forces in molecules}, R.P. Feynman, Physical Review \textbf{56}, 340, (1939).

%\bibitem{anderson} \textit{The resonating valence bond state in La$_2$CuO$_4$ and superconductivity} Science \textbf{235}, 1196, (1987).


%\end{thebibliography}



\end{document}
